Co-designing efficient machine learning based systems across
the whole hardware/software stack to trade off speed, accuracy, energy
and costs is becoming extremely complex and time consuming.
%
Researchers often struggle to evaluate and compare different
published works across rapidly evolving software
frameworks, heterogeneous hardware platforms, compilers,
libraries, algorithms, data sets, models, and environments.

We~\footnote{ReQuEST organizers (A-Z): 
Luis Ceze, University of Washington (USA),
Natalie Enright Jerger (University of Toronto, Canada),
Babak Falsafi (EPFL, Switzerland),
Grigori Fursin, (cTuning foundation, France),
Anton Lokhmotov, (dividiti, UK),
Thierry Moreau, (University of Washington, USA),
Adrian Sampson, (Cornell University, USA)
Phillip Stanley Marbell, (University of Cambridge, UK)}
present our community effort to develop an open co-design tournament 
platform with an online public scoreboard.
%
It will gradually incorporate best research practices 
while providing a common way for multidisciplinary researchers
to optimize and compare the quality vs. efficiency Pareto
optimality of various workloads on diverse and complete
hardware/software systems.
%
We want to leverage the open-source Collective Knowledge framework
and the ACM artifact evaluation methodology to validate and share the complete 
machine learning system implementations
in a standardized, portable, and reproducible fashion.
%
We plan to hold regular multi-objective optimization and co-design tournaments 
for emerging workloads such as deep learning, starting with ASPLOS'18
(ACM conference on Architectural Support for Programming Languages and Operating Systems 
- the premier forum for multidisciplinary systems research spanning computer architecture 
and hardware, programming languages and compilers, operating systems and networking)
to build a public repository of the most efficient algorithms and systems
which can be easily reused and built upon.
%
We will also use the feedback from participants to continue improving
our platform and common co-design methodology.

